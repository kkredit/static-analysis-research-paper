%%%%%%%%%%%%%%%%%%%%%%%%%%%%%%%%%%%%%%%%%%%%%%%%%%%%%%%%%%%%%%%%%%%%%%% SETUP %
\documentclass[journal]{IEEEtran}
\usepackage{cite}
\usepackage{amsmath,amssymb,amsfonts}
\usepackage{algorithmic}
\usepackage{graphicx}
\usepackage{textcomp}
\usepackage{xcolor}
\def\BibTeX{{\rm B\kern-.05em{\sc i\kern-.025em b}\kern-.08em
    T\kern-.1667em\lower.7ex\hbox{E}\kern-.125emX}}

\begin{document}


%%%%%%%%%%%%%%%%%%%%%%%%%%%%%%%%%%%%%%%%%%%%%%%%%%%%%%%%%%%%%%%%%%%%%%% TITLE %

\title{FlawFinder and Secure Development Processes}

\author{\IEEEauthorblockN{Kevin Kredit}
\IEEEauthorblockA{\textit{GVSU CIS Department} \\
Grand Rapids, Michigan \\
Email: k.kredit.us@ieee.org}
}

\maketitle


%%%%%%%%%%%%%%%%%%%%%%%%%%%%%%%%%%%%%%%%%%%%%%%%%%%%%%%%%%%%%%%%%%%%%%% ABSTRACT %

\begin{abstract}
Static analysis tools provide insights to security and correctness early in the software development
process. FlawFinder is a static analysis tool for C and C++ focusing on dangerous method use. It
performs a syntax-aware search, identifies dangerous functions uses, and prints a ranked list of
warnings. A complete secure development process may employ multiple static analysis tools. Statick
is a plugin-based framework to manage multiple tools, including FlawFinder. To simplify the workflow
for small projects, a Docker container has been built with a portable Statick environment.
\end{abstract}

\begin{IEEEkeywords}
FlawFinder, Statick, static analysis, secure software, development process
\end{IEEEkeywords}


%%%%%%%%%%%%%%%%%%%%%%%%%%%%%%%%%%%%%%%%%%%%%%%%%%%%%%%%%%%%%%%%%%%%%%% PAPER %

\section{Introduction}

Security cannot be bolted on. The sooner project developers consider security, the cheaper it is to
achieve. If a project can continuously validate application security principles before it is
finished, or even before it builds, it will experience fewer costly surprises later on.

Static analysis tools provide this continuous validation capability. They read source code and
report possible errors in a developer-friendly format. Other validation techniques such as
functional tests, unit tests, binary analysis, and runtime analysis require varying levels of
readiness. Because static analysis tools require only source code, developers can integrate them
from the very beginning of the process.

FlawFinder is a static analysis tool for C and C++ \cite{flawfinder}. It works by searching for
invocations of dangerous or hard to use functions, ranking them based in order of risk, and
displaying them with helpful messages.

This paper gives a brief overview of static analysis tools in general, including their purpose,
their capabilities and limitations, and how they fit in a secure software engineering workflow. It
then describes FlawFinder in particular, including how it works, how it is configured, and example
output from running it against a real project. Finally, it strives to lower the barrier for static
analysis tool usage by describing the Statick framework and introducing a project that provides the
Statick environment in a portable Docker container.

\subsection{Static Analysis Tools}
% A little intro to static analysis tools.
Two functions: correct basic human errors; apply logic to provide insights that surpass human
capabilities.

Humans have a tendency to be imperfect, making basic errors, using dangerous constructs, and
violating coding standards. Several of these errors are relatively simple to detect via automated
source analysis tools.

Static analysis tools exist to correct these little errors.

\subsection{FlawFinder}
A little intro to FlawFinder.

\subsection{This Paper's Contribution}
What this paper will have. (How to make it easier!)


\section{Static Analysis Tools}

\subsection{Secure Software Development Process}

\subsection{Capabilities and Limitations}


\section{FlawFinder}

\subsection{Purpose}

\subsection{How It Works}

\subsection{Example}


\section{Easy Static Analysis}

\subsection{Lower the Barrier to Entry}

\subsection{Frameworks}

\subsection{Docker Container}


\section{Conclusion}
Conclusion.


%%%%%%%%%%%%%%%%%%%%%%%%%%%%%%%%%%%%%%%%%%%%%%%%%%%%%%%%%%%%%%%%%%%%%%% BIBLIOGRAPHY %

\begin{thebibliography}{00}

\bibitem{flawfinder} D. Wheeler, "Flawfinder Home Page", Dwheeler.com, 2019. [Online]. Available:
https://dwheeler.com/flawfinder/. [Accessed: 02- Nov- 2019].

\bibitem{statick} "sscpac/statick", GitHub, 2019. [Online]. Available:
https://github.com/sscpac/statick. [Accessed: 02- Nov- 2019].

\end{thebibliography}


\end{document}


%%%%%%%%%%%%%%%%%%%%%%%%%%%%%%%%%%%%%%%%%%%%%%%%%%%%%%%%%%%%%%%%%%%%%%% REFERENCE %

% \begin{equation}
% a+b=\gamma\label{eq}
% \end{equation}

% \paragraph{A labeled paragraph} With some text.

% \begin{table}[htbp]
% \caption{Table Type Styles}
% \begin{center}
% \begin{tabular}{|c|c|c|c|}
% \hline
% \textbf{Table}&\multicolumn{3}{|c|}{\textbf{Table Column Head}} \\
% \cline{2-4}
% \textbf{Head} & \textbf{\textit{Table column subhead}}& \textbf{\textit{Subhead}}&
%\textbf{\textit{Subhead}} \\
% \hline
% copy& More table copy$^{\mathrm{a}}$& &  \\
% \hline
% \multicolumn{4}{l}{$^{\mathrm{a}}$Sample of a Table footnote.}
% \end{tabular}
% \label{tab1}
% \end{center}
% \end{table}

% \begin{figure}[htbp]
% \centerline{\includegraphics{fig1.png}}
% \caption{Example of a figure caption.}
% \label{fig}
% \end{figure}
